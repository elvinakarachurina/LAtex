\documentclass[a4paper,12pt]{book}[2018/09/03]
\usepackage[utf8]{inputenc}
\usepackage[english,russian]{babel}
\usepackage{indentfirst}
\usepackage{misccorr}
\usepackage{graphicx}
\usepackage{amsmath}
\begin{document}
\textbf{9.3 Умножение}

Из теорем 9.1 и 9.3 легко следует,что существует S, которая умножает n-разрядных двоичных числа со сложностью O(n$^{2})$ и глубиной O($log_{2}$n). Покажем, что существуют более простые схемы. Использованная в следующей теореме конструкция принадлежит А. А. Карацубе, который первым показал, что два n-разрядных числа можно умножить со сложностью меньшей чем n$^{2}$.
\parskip=3mm

\textbf{Теорема 9.4.}
\textit{Существует схема $M_{n}$, вычисляющая произведение двух n-разрядных целых чисел, для сложности и глубины которой при n $\to $ $\infty $ справедливы неравенства}
\begin{center}
L($M_{n}$)=O(n$^{log_{2}3}$), \quad{ D($M_{n}$)=O$({log_{2}n}).$}
\end{center}
\textsc{Доказательство.}
Пусть n = 2k + 2, x и y — произвольные (2n − 2)-разрядные двойные числа. Представим их в виде
\begin{center}
x=$x_{2}$2$^{n-1}$+$x_{1}$, \quad{y=$y_{2}$2$^{n-1}$+$y_{1},$}
\end{center}
где каждое из чисел $x_{1}$, $x_{2}$, $y_{1}$, $y_{2}$ состоит не более чем из n - 1 разрядов.
Тогда
\begin{center}
xy=$x_{2}$$y_{2}$2$^{n-1}$+($x_{2}$$y_{1}$+$x_{1}$$y_{2}$)2$^{n-1}$+$x_{1}$$y_{1}$.
\end{center}
Откуда после несложных преобразований для произведения xy получаем равенство
\begin{center}
xy=$x_{2}$$y_{2}$2$^{n-2}$+($x_{2}$$y_{2}$+$x_{1}$$y_{1}$)2$^{n-1}$-($x_{2}$ - $x_{1}$)($y_{2}$ - $y_{1}$)2$^{n-1}$+$x_{1}$$y_{1}$. (9.5)
\end{center}
Следовательно, умножение двух (2n - 2)-разрядных чисел сводится к двум умножениям (n-1)-разрядных чисел, одному умножению n-разрядных чисел и нескольким сложениям.

Рекурсивная конструкция схемы $M_{2n-2}$ показана на рисунке 9.4, где
\begin{flushleft}
$z_{1}$=$x_{2}$$y_{2}$+$x_{1}$$y_{1}$

$z_{2}$=($x_{2}$ - $x_{1}$)($y_{2}$ - $y_{1}$)

$z_{3}$=$x_{2}$$y_{2}$2$^{n-2}$+$z_{1}$2$^{n-1}$+$x_{1}$$y_{1}$
\end{flushleft}
Нетрудно видеть, что для сложности и глубины схемы $M_{2n-2}$ имеют место рекуррентные соотношения
\begin{center}
L($M_{2n-2}$) \le 3L$M_{n}$+O(n), \quad D($M_{2n-2}$) \le D($M_{n}$)+O(n).
\end{center}
Так как 2n - 2=2$^{k+1}$ + 2 и n = 2$^{k}$ + 2, то последние неравенства можно переписать в виде
\begin{center}
L($M_{2^{k+1}+{2}}$) \le 3L($M_{2^{k}+{2}}$)+O(2$^{k}$), \quad D($M_{2^{k+1}+{2}}$) \le  D($M_{2^{k}+{2}}$)+O(1)
\end{center}
Тогда справедливы неравенства
\begin{center}
L($M_{2^{k}+{2}}$) \le  O($2^{klog_{2}3}$), \quad D($M_{2^{k}+{2}}$) \le O(k),
\end{center}
из которых легко следует утверждение теоремы. Теорема доказана.

В настоящее время разработаны различные алгоритмы умножения целых n-разрядных чисел, позволяющие строить при больших n значительно более экономные схемы. Наиболее простые из этих схем (см. [39]) состоят из O(n$log_2n$ · $log_{2}$$log_{2}$n) элементов, а их глубина равна O($log_{2}$n). К сожалению константы, входящие в ”O” таковы, что эти схемы представляют в основном теоретический интерес.

\end{document}